\Transcb{yellow}{blue}{Exercises}
{\red
\begin{itemize}
\item Consider the Westerbork radio telescope at
      $52^{\circ}\,54^{\prime}\,54.33^{\prime\prime}$N
      $6^{\circ}\,36^{\prime}\,12.74^{\prime\prime}$E.
\item[] At 06-sep-2011 21:10:34.7 UTC one wants to observe the Andromeda galaxy.
\item[] Andromeda galaxy (M31) : $\alpha=0\text{h}\,42.7\text{m} \quad \delta=41^{\circ}\,16^{\prime}$ (J2000)
\item[$\ast$] What are the horizontal coordinates to aim the telescope at ?
\item Consider the IceCube experiment at the South Pole.
\item[] The experiment uses the following righthanded local coordinate system :
\item[] Z-axis points to Zenith, Y-axis points North, X-axis points East.
\item[] At 15-aug-2009 06:23:16.2 UTC we observed a very energetic muon track.
\item[] Track direction $\theta=12^{\circ} \quad \phi=138^{\circ}$
\item[$\ast$] Provide the Equatorial and Galactic coordinates of the source.
\item[$\ast$] Show the location of the source on an Equatorial and Galactic skymap.
\item[] Hint : Use NcAstrolab
\end{itemize}
}

\Tr
{\red
\begin{itemize}
\item Cosmic rays impinge on the Earth's atmosphere from all directions.
\item[] The resulting angular distribution is isotropic.
\item[$\ast$] Generate arrival directions for 1000 cosmic rays and use NcAstrolab
\item[] to show that the observed angular distribution is indeed isotropic.
\end{itemize}
}