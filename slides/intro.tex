\Transcb{yellow}{blue}{Introduction}
\begin{itemize}
\item Exploring the Universe $\rightarrow$ {\blue Need for a coordinate system}
\item But {\blue everything is moving} $\rightarrow$ Complicated situation
\begin{itemize}
\item Earth spins around its axis
\item Influence of Sun, Moon, Planets $\rightarrow$ Earth axis subjected to precession
\item Earth axis also subjected to nutation
\item Earth rotates around the Sun
\item Sun rotates around the center of our Galaxy
\item Our Galaxy moves through space within the local cluster
\end{itemize}
\item What should we take as origin and orientation of the axes ?
\item[$\ast$] \colorbox{yellow}{Use different coordinate systems depending on what one wants to observe}
\item[] Okido, we are used to that (e.g. Cartesian, Spherical, Comoving, ...)
\item Due to the movements we {\blue need also a time system}
\item[] Effects of the above movements are observed on very different timescales
\item[$\ast$] \colorbox{yellow}{Use different time systems depending on what one wants to observe}
\item[] Not always practical (a lab with many clocks) $\rightarrow$ {\blue Overall time ?}
\end{itemize}
